% ------------------------------------------------------------------------------------------------ %
%
% Важно: для компиляции используется строго "xelatex" !!!
%
% ------------------------------------------------------------------------------------------------ %
% Настройка параметров документа


%% Use the "normalphoto" option if you want a normal photo instead of cropped to a circle
% \documentclass[10pt,a4paper,normalphoto]{altacv}

\documentclass[10pt,a4paper,ragged2e,withhyper]{altacv}
%% AltaCV uses the fontawesome5 and simpleicons packages.
%% See http://texdoc.net/pkg/fontawesome5 and http://texdoc.net/pkg/simpleicons for full list of symbols.

% Change the page layout if you need to
\geometry{left=1.25cm, right=1.25cm, top=1.5cm, bottom=1.5cm, columnsep=1.2cm}

% Пакет paracol позволяет набирать столбцы текста параллельно (т.е. бок о бок).
\usepackage{paracol}


% ------------------------------------------------------------------------------------------------ %
% Подключение файлов с настройками

% ------------------------------------------------------------------------------------------------ %
%
% Файл с настройкой шрифтов для документа
% Шрифты берутся из отдельной директории
%
% ------------------------------------------------------------------------------------------------ %

% Пакет для работы со шрифтами
\usepackage{fontspec}

% Команды-переменные для хранения названия единственного шрифта для всего документа

\newcommand{\varFontsGlobal}{PT Sans}
\newcommand{\varFontsGlobalSystemName}{PT_Sans-Web}

% Команды-переменные для основного шрифта для всего документа: латиница, цифры, символы

\newcommand{\varFontsMain}{\varFontsGlobal}
\newcommand{\varFontsMainSystemName}{\varFontsGlobalSystemName}

% Команды-переменные для шрифта без засечек

\newcommand{\varFontsSans}{\varFontsGlobal}
\newcommand{\varFontsSansSystemName}{\varFontsGlobalSystemName}

% Команды-переменные для символов кириллицы

\newcommand{\varFontsCyrillic}{\varFontsGlobal}
\newcommand{\varFontsCyrillicSystemName}{\varFontsGlobalSystemName}

% Команды-переменные для файлов шрифтов

\newcommand{\varFontsDirectory}{../../fonts/}
\newcommand{\varFontsExtension}{.ttf}


% Команда-функция настройки основного шрифта
\newcommand{\funcFontsSetMain}[4]
{
  % Установка основного шрифта
  \setmainfont{#1}                % Название основного шрифта (может не совпадать с именем в файловой системе) (параметр 1)
  [
    Path=#2,                      % Путь до директории файла шрифта (параметр 2)
    Extension=#3,                 % Расширение файла шрифта с точкой (параметр 3)
    UprightFont=#4,               % Имя шрифта в файловой системе без расширения (параметр 4)
    BoldFont=#4-Bold,             % Имя жирного шрифта
    ItalicFont=#4-Italic,         % Имя наклонного шрифта
    BoldItalicFont=#4-BoldItalic  % Имя жирно-наклонного шрифта
  ]
}

% Команда-функция настройки шрифта без засечек
\newcommand{\funcFontsSetSans}[4]
{
  % Установка основного шрифта без засечек
  \setsansfont{#1}                % Название основного шрифта (может не совпадать с именем в файловой системе) (параметр 1)
  [
    Path=#2,                      % Путь до директории файла шрифта (параметр 2)
    Extension=#3,                 % Расширение файла шрифта с точкой (параметр 3)
    UprightFont=#4,               % Имя шрифта в файловой системе без расширения (параметр 4)
    BoldFont=#4-Bold,             % Имя жирного шрифта
    ItalicFont=#4-Italic,         % Имя наклонного шрифта
    BoldItalicFont=#4-BoldItalic  % Имя жирно-наклонного шрифта
  ]
}

% Команда-функция настройки шрифта кириллицы
\newcommand{\funcFontsSetCyrillic}[4]
{
  % Установка основного шрифта для кириллицы
  \newfontfamily\cyrillicfont{#1} % Название основного шрифта (может не совпадать с именем в файловой системе) (параметр 1)
  [
    Path=#2,                      % Путь до директории файла шрифта (параметр 2)
    Extension=#3,                 % Расширение файла шрифта с точкой (параметр 3)
    UprightFont=#4,               % Имя шрифта в файловой системе без расширения (параметр 4)
    BoldFont=#4-Bold,             % Имя жирного шрифта
    ItalicFont=#4-Italic,         % Имя наклонного шрифта
    BoldItalicFont=#4-BoldItalic  % Имя жирно-наклонного шрифта
  ]
}

% Основная команда-функция для настройки всех шрифтов документа
\newcommand{\funcFontsSet}
{
  \usepackage{polyglossia}

  \setmainlanguage{russian}
  \setotherlanguage{english}

  % Настройка шрифтов: основной, без засечек и кириллицу

  \funcFontsSetMain{\varFontsMain}{\varFontsDirectory}{\varFontsExtension}{\varFontsMainSystemName}
  \funcFontsSetSans{\varFontsSans}{\varFontsDirectory}{\varFontsExtension}{\varFontsSansSystemName}
  \funcFontsSetCyrillic{\varFontsCyrillic}{\varFontsDirectory}{\varFontsExtension}{\varFontsCyrillicSystemName}

  % Выбор шрифта по умолчанию:
  %  - \rmdefault - с засечками (serif)         - Roman
  %  - \sfdefault - без засечек (sans-serif)    - Sans-serif
  %  - \ttdefault - моноширинный (typewriter)   - Teletype

  \renewcommand{\familydefault}{\rmdefault}

  % Настройка размера шрифтов для конкретных полей
  %
  % | Команда         | Относительный размер | Описание                                |
  % | --------------- | -------------------- | --------------------------------------- |
  % | \tiny           | Очень маленький      | Самый мелкий размер                     |
  % | \scriptsize     | Очень маленький      | Чаще всего для подписей под формулами   |
  % | \footnotesize   | Маленький            | Для сносок, подвалов, мелких деталей    |
  % | \small          | Чуть меньше обычного | Используется для второстепенного текста |
  % | \normalsize     | Обычный размер       | Размер по умолчанию                     |
  % | \large          | Чуть больше обычного | Для выделения, но не заголовков         |
  % | \Large          | Ещё больше           | Часто используется в заголовках         |
  % | \LARGE          | Большой              | Подзаголовки или крупный текст          |
  % | \huge           | Очень большой        | Реже используется, например в титуле    |
  % | \Huge           | Самый большой        | Максимальный стандартный размер         |

  % Настройка стиля шрифтов
  %
  % | Команда     | Что делает (весь текст после неё) |
  % | ----------- | --------------------------------- |
  % | \rmfamily   | Шрифт с засечками                 |
  % | \sffamily   | Без засечек                       |
  % | \ttfamily   | Моноширинный                      |
  % | \bfseries   | Жирный                            |
  % | \mdseries   | Обычный вес (medium)              |
  % | \itshape    | Курсив                            |
  % | \slshape    | Наклонный (slanted)               |
  % | \scshape    | Малые заглавные                   |
  % | \upshape    | Вертикальный, не наклонный текст  |


  \renewcommand{\namefont}        {\huge\rmfamily\bfseries}   % Имя кандидата
  \renewcommand{\personalinfofont}{\footnotesize}             % Инфо кандидата: телефон, почта и т.д.
  \renewcommand{\cvsectionfont}   {\LARGE\rmfamily\bfseries}  % Заголовки секций резюме
  \renewcommand{\cvsubsectionfont}{\large\bfseries}           %
}
    % Подключение файла с настройками шрифтов
% ------------------------------------------------------------------------------------------------ %
% Файл с настройкой цветовой гаммы для документа
%
% ------------------------------------------------------------------------------------------------ %


% Настройка псевдонимов для нестандартных цветов

\definecolor{SlateGrey}{HTML}{2E2E2E}
\definecolor{LightGrey}{HTML}{666666}
\definecolor{DarkPastelRed}{HTML}{450808}
\definecolor{PastelRed}{HTML}{8F0D0D}
\definecolor{GoldenEarth}{HTML}{E7D192}


% Настройка цветов для конкретных секций
\newcommand{\funcColorsSet}
{
  \colorlet{name}       {black}         % Имя кандидата
  \colorlet{tagline}    {PastelRed}     % Должность или краткое описание
  \colorlet{heading}    {DarkPastelRed} %
  \colorlet{headingrule}{GoldenEarth}   %
  \colorlet{subheading} {PastelRed}     %
  \colorlet{accent}     {PastelRed}     %
  \colorlet{emphasis}   {SlateGrey}     %
  \colorlet{body}       {LightGrey}     %
}   % Подключение файла с настройками цветов
% ------------------------------------------------------------------------------------------------ %
% Файл с настройкой цветовой гаммы для документа
%
% ------------------------------------------------------------------------------------------------ %


% Описание информации о кандидате

\renewcommand{\varInfoName}{Человечек паучок}

     % Подключение файла с личной информацией


% ------------------------------------------------------------------------------------------------ %
% Настройка шрифтов для всего документа из файла settings/fonts.tex

\funcFontsSet


% ------------------------------------------------------------------------------------------------ %
% Настройка цветовой гаммы для всего документа из файла settings/colors.tex

\funcColorsSet



% Change the bullets for itemize and rating marker
% for \cvskill if you want to
\renewcommand{\cvItemMarker}{{\small\textbullet}}
\renewcommand{\cvRatingMarker}{\faCircle}
% ...and the markers for the date/location for \cvevent
% \renewcommand{\cvDateMarker}{\faCalendar*[regular]}
% \renewcommand{\cvLocationMarker}{\faMapMarker*}


% If your CV/résumé is in a language other than English,
% then you probably want to change these so that when you
% copy-paste from the PDF or run pdftotext, the location
% and date marker icons for \cvevent will paste as correct
% translations. For example Spanish:
% \renewcommand{\locationname}{Ubicación}
% \renewcommand{\datename}{Fecha}


%% Use (and optionally edit if necessary) this .tex if you
%% want to use an author-year reference style like APA(6)
%% for your publication list
% % When using APA6 if you need more author names to be listed
% because you're e.g. the 12th author, add apamaxprtauth=12
\usepackage[backend=biber,style=apa6,sorting=ydnt]{biblatex}
\defbibheading{pubtype}{\cvsubsection{#1}}
\renewcommand{\bibsetup}{\vspace*{-\baselineskip}}
\AtEveryBibitem{%
  \makebox[\bibhang][l]{\itemmarker}%
  \iffieldundef{doi}{}{\clearfield{url}}%
}
\setlength{\bibitemsep}{0.25\baselineskip}
\setlength{\bibhang}{1.25em}


%% Use (and optionally edit if necessary) this .tex if you
%% want an originally numerical reference style like IEEE
%% for your publication list
\usepackage[backend=biber,style=ieee,sorting=ydnt,defernumbers=true]{biblatex}
%% For removing numbering entirely when using a numeric style
\setlength{\bibhang}{1.25em}
\DeclareFieldFormat{labelnumberwidth}{\makebox[\bibhang][l]{\itemmarker}}
\setlength{\biblabelsep}{0pt}
\defbibheading{pubtype}{\cvsubsection{#1}}
\renewcommand{\bibsetup}{\vspace*{-\baselineskip}}
\AtEveryBibitem{%
  \iffieldundef{doi}{}{\clearfield{url}}%
}


%% sample.bib contains your publications
\addbibresource{sample.bib}

\begin{document}

    % Генерация блоков резюме по отдельности

    % -------------------------------------------------------------------------------------------- %
    % 1. Заголовок с личной информацией - Имя, должность, блок информации и фотографии
    %    Берется из файла settings/info.tex

    \funcInfoSetName        % Первая строка - имя кандидата
    \funcInfoSetPosition    % Вторая строка - должность(и)
    \funcInfoSetPhotos      % Фотографии - могут быть две: слева и/или справа от блока информации
    \funcInfoSetTopBlock    % Блок информации под должностью и именем

    \makecvheader           % Функция генерации заголовка: без нее заголовка не будет вообще

    %% Depending on your tastes, you may want to make fonts of itemize environments slightly smaller
    \AtBeginEnvironment{itemize}{\small}

    %% Set the left/right column width ratio to 6:4.
    \columnratio{0.6}

    % Start a 2-column paracol. Both the left and right columns will automatically
    % break across pages if things get too long.
    \begin{paracol}{2}

        \cvsection{Experience}

        \cvevent{Job Title 1}{Company 1}{Month 20XX -- Ongoing}{Location}
        \begin{itemize}
        \item Job description 1
        \item Job description 2
        \end{itemize}

        \divider

        \cvevent{Job Title 2}{Company 2}{Month 20XX -- Ongoing}{Location}
        \begin{itemize}
        \item Job description 1
        \item Job description 2
        \end{itemize}

        \cvsection{Projects}

        \cvevent{Project 1}{Funding agency/institution}{}{}
        \begin{itemize}
        \item Details
        \end{itemize}

        \divider

        \cvevent{Project 2}{Funding agency/institution}{Project duration}{}
        A short abstract would also work.

        \medskip

        \cvsection{A Day of My Life}

        % Adapted from @Jake's answer from http://tex.stackexchange.com/a/82729/226
        % \wheelchart{outer radius}{inner radius}{
        % comma-separated list of value/text width/color/detail}
        \wheelchart{1.5cm}{0.5cm}{%
        6/8em/accent!30/{Sleep,\\beautiful sleep},
        3/8em/accent!40/Hopeful novelist by night,
        8/8em/accent!60/Daytime job,
        2/10em/accent/Sports and relaxation,
        5/6em/accent!20/Spending time with family
        }

        % use ONLY \newpage if you want to force a page break for
        % ONLY the current column
        \newpage

        \cvsection{Publications}

        %% Specify your last name(s) and first name(s) as given in the .bib to automatically bold your own name in the publications list.
        %% One caveat: You need to write \bibnamedelima where there's a space in your name for this to work properly; or write \bibnamedelimi if you use initials in the .bib
        %% You can specify multiple names, especially if you have changed your name or if you need to highlight multiple authors.
        \mynames{Lim/Lian\bibnamedelima Tze,
        Wong/Lian\bibnamedelima Tze,
        Lim/Tracy,
        Lim/L.\bibnamedelimi T.}
        %% MAKE SURE THERE IS NO SPACE AFTER THE FINAL NAME IN YOUR \mynames LIST

        \nocite{*}

        \printbibliography[heading=pubtype,title={\printinfo{\faBook}{Books}},type=book]

        \divider

        \printbibliography[heading=pubtype,title={\printinfo{\faFile*[regular]}{Journal Articles}},type=article]

        \divider

        \printbibliography[heading=pubtype,title={\printinfo{\faUsers}{Conference Proceedings}},type=inproceedings]

        %% Switch to the right column. This will now automatically move to the second
        %% page if the content is too long.
        \switchcolumn

        \cvsection{My Life Philosophy}

        \begin{quote}
        ``Something smart or heartfelt, preferably in one sentence.''
        \end{quote}

        \cvsection{Most Proud of}

        \cvachievement{\faTrophy}{Fantastic Achievement}{and some details about it}

        \divider

        \cvachievement{\faHeartbeat}{Another achievement}{more details about it of course}

        \divider

        \cvachievement{\faHeartbeat}{Another achievement}{more details about it of course}

        \cvsection{Strengths}

        % Don't overuse these \cvtag boxes — they're just eye-candies and not essential. If something doesn't fit on a single line, it probably works better as part of an itemized list (probably inlined itemized list), or just as a comma-separated list of strengths.

        \cvtag{Hard-working}
        \cvtag{Eye for detail}\\
        \cvtag{Motivator \& Leader}

        \divider\smallskip

        \cvtag{C++}
        \cvtag{Embedded Systems}\\
        \cvtag{Statistical Analysis}

        \cvsection{Languages}

        \cvskill{English}{5}
        \divider

        \cvskill{Spanish}{4}
        \divider

        \cvskill{German}{3.5} %% Supports X.5 values.

        %% Yeah I didn't spend too much time making all the
        %% spacing consistent... sorry. Use \smallskip, \medskip,
        %% \bigskip, \vspace etc to make adjustments.
        \medskip

        \cvsection{Education}

        \cvevent{Ph.D.\ in Your Discipline}{Your University}{Sept 2002 -- June 2006}{}
        Thesis title: Wonderful Research

        \divider

        \cvevent{M.Sc.\ in Your Discipline}{Your University}{Sept 2001 -- June 2002}{}

        \divider

        \cvevent{B.Sc.\ in Your Discipline}{Stanford University}{Sept 1998 -- June 2001}{}

        % \divider

        \cvsection{Referees}

        % \cvref{name}{email}{mailing address}
        \cvref{Prof.\ Alpha Beta}{Institute}{a.beta@university.edu}
        {Address Line 1\\Address line 2}

        \divider

        \cvref{Prof.\ Gamma Delta}{Institute}{g.delta@university.edu}
        {Address Line 1\\Address line 2}


    \end{paracol}


\end{document}
