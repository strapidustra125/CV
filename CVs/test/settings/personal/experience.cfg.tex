% ------------------------------------------------------------------------------------------------ %
%
% Файл с описанием опыта
%
% ------------------------------------------------------------------------------------------------ %

\newcommand{\varExperienceJobUrlSez}            {https://zaprib.ru/}
\newcommand{\varExperienceJobUrlSsu}            {https://www.sgu.ru/}
\newcommand{\varExperienceJobUrlTtm}            {https://transtelematica.ru/}
\newcommand{\varExperienceJobUrlTtmMnt}         {https://transtelematica.ru/projects/programmnoe-obespechenie/programmnoe-obespechenie-MNT/}

% Даты работы в формате ГГГГ-ММ-ДД

\newcommand{\varExperienceJobTtmSeniorFinish}   {}
\newcommand{\varExperienceJobTtmSeniorStart}    {2024-09-01}
\newcommand{\varExperienceJobTtmMiddleFinish}   {2024-09-01}
\newcommand{\varExperienceJobTtmMiddleStart}    {2023-04-01}
\newcommand{\varExperienceJobTtmJuniorFinish}   {2023-04-01}
\newcommand{\varExperienceJobTtmJuniorStart}    {2021-10-18}

\newcommand{\varExperienceJobSsuFinish}         {2025-01-01}
\newcommand{\varExperienceJobSsuStart}          {2023-01-01}

\newcommand{\varExperienceJobSezFinish}         {2021-10-18}
\newcommand{\varExperienceJobSezStart}          {2019-10-25}

% Добавить функцию перевода даты ГГГГ-ММ-ДД в ММММ ГГГГ



\usepackage{expl3,xparse}

\ExplSyntaxOn


\seq_new:N \g_months_russian_seq
\seq_set_from_clist:Nn \g_months_russian_seq
  {Январь,Февраль,Март,Апрель,Май,Июнь,Июль,Август,Сентябрь,Октябрь,Ноябрь,Декабрь}

\NewDocumentCommand{\monthyearfromdate}{m}
 {
   \monthyearfromdate:n {#1}
 }

\cs_new_protected:Npn \monthyearfromdate:n #1
 {
   % Если длина не 10 — сразу жирным и без заморочек
   \int_compare:nNnTF { \str_count:n {#1} } = { 10 }
     { \monthyearfromdate_try_parse:n {#1} }
     { \monthyearfromdate_print_bold:n {#1} }
 }

\cs_new_protected:Npn \monthyearfromdate_try_parse:n #1
 {
   % Попытка сплитнуть, без проверки дефисов, просто если не формат — упадём в жирный
   \seq_set_split:Nnn \l_tmpa_seq { - } { #1 }
   \int_compare:nNnTF { \seq_count:N \l_tmpa_seq } = { 3 }
     {
       \seq_pop_left:NN \l_tmpa_seq \l_tmpa_tl % год
       \seq_pop_left:NN \l_tmpa_seq \l_tmpb_tl % месяц
       % Проверяем месяц 1..12
       \int_set:Nn \l_tmpa_int { \tl_use:N \l_tmpb_tl }
       \int_compare:nNnTF { \l_tmpa_int } > { 0 }
         {
           \int_compare:nNnTF { \l_tmpa_int } < { 13 }
             { \monthyearfromdate_parse_date:nn { \l_tmpa_tl } { \l_tmpb_tl } }
             { \monthyearfromdate_print_bold:n {#1} }
         }
         { \monthyearfromdate_print_bold:n {#1} }
     }
     { \monthyearfromdate_print_bold:n {#1} }
 }

\cs_new_protected:Npn \monthyearfromdate_parse_date:nn #1 #2
 {
   \seq_item:Nn \g_months_russian_seq { \int_eval:n {#2} }~#1
 }

\cs_new_protected:Npn \monthyearfromdate_print_bold:n #1
 {
   \textbf{#1}
 }

% ------------------------------------------------------------------------------------------------ %

% Склонения лет
\cs_new:Npn \get_year_word:n #1
  {
    \int_case:nnF { \int_mod:nn {#1} {100} }
      {
        {11} {лет}
        {12} {лет}
        {13} {лет}
        {14} {лет}
      }
      {
        \int_case:nnF { \int_mod:nn {#1} {10} }
          {
            {1} {год}
            {2} {года}
            {3} {года}
            {4} {года}
          }
          {лет}
      }
  }

% Склонения месяцев
\cs_new:Npn \get_month_word:n #1
  {
    \int_case:nnF { \int_mod:nn {#1} {100} }
      {
        {11} {месяцев}
        {12} {месяцев}
        {13} {месяцев}
        {14} {месяцев}
      }
      {
        \int_case:nnF { \int_mod:nn {#1} {10} }
          {
            {1} {месяц}
            {2} {месяца}
            {3} {месяца}
            {4} {месяца}
          }
          {месяцев}
      }
  }

% Главная функция: разность текущей даты и заданной
\NewDocumentCommand{\datediff}{mmmmmm}
 {
   % #1-год1, #2-месяц1, #3-день1
   % #4-год2, #5-месяц2, #6-день2
   \int_compare:nNnTF { #4 } < { #1 }
     { \textbf{Error: Некорректный порядок дат} }
     {
       \int_set:Nn \l_tmpa_int { #4 - #1 } % годы
       \int_set:Nn \l_tmpb_int { #5 - #2 } % месяцы

       \int_compare:nNnTF { \l_tmpb_int } < { 0 }
         {
           \int_decr:N \l_tmpa_int
           \int_add:Nn \l_tmpb_int { 12 }
         }
         {}

       % Вывод лет
       \int_to_arabic:n { \l_tmpa_int }~\get_year_word:n { \l_tmpa_int }

       % Вывод месяцев (только если не 0)
       \int_compare:nNnF { \l_tmpb_int } = { 0 }
         {
           ,~\int_to_arabic:n { \l_tmpb_int }~\get_month_word:n { \l_tmpb_int }
         }
     }
 }

 % Промежуточная обёртка: \datediffstr{1990-12-25}
\NewDocumentCommand{\datediffstr}{mm}
 {
   \datediffstr_process:nn {#1} {#2}
 }

\cs_new_protected:Npn \datediffstr_process:nn #1#2
 {
   % Разбираем первую дату (всегда предполагается в формате YYYY-MM-DD)
   \seq_set_split:Nnn \l_tmpa_seq { - } { #1 }
   \seq_pop_left:NN \l_tmpa_seq \l_tmpa_tl % год1
   \seq_pop_left:NN \l_tmpa_seq \l_tmpb_tl % месяц1
   \seq_pop_left:NN \l_tmpa_seq \l_tmpc_tl % день1

   % Проверяем формат второй даты
   \str_if_eq:nnTF { \tl_count:n {#2} } {10}
     {
       \str_if_eq:nnTF { \tl_item:nn {#2} {5} } {-}
         {
           \str_if_eq:nnTF { \tl_item:nn {#2} {8} } {-}
             {
               % Всё норм — разбираем вторую дату
               \seq_set_split:Nnn \l_tmpb_seq { - } { #2 }
               \seq_pop_left:NN \l_tmpb_seq \l_tmpd_tl % год2
               \seq_pop_left:NN \l_tmpb_seq \l_tmpe_tl % месяц2
               \seq_pop_left:NN \l_tmpb_seq \l_tmpf_tl % день2
             }
             {
               \datediffstr_use_today:
             }
         }
         {
           \datediffstr_use_today:
         }
     }
     {
       \datediffstr_use_today:
     }

   % Вызываем основную функцию
   \datediff
     { \tl_use:N \l_tmpa_tl }
     { \tl_use:N \l_tmpb_tl }
     { \tl_use:N \l_tmpc_tl }
     { \tl_use:N \l_tmpd_tl }
     { \tl_use:N \l_tmpe_tl }
     { \tl_use:N \l_tmpf_tl }
 }

% Если вторая дата невалидна — подставляем текущую
\cs_new_protected:Npn \datediffstr_use_today:
 {
   \tl_set:Nx \l_tmpd_tl { \int_eval:n { \year } }
   \tl_set:Nx \l_tmpe_tl { \int_eval:n { \month } }
   \tl_set:Nx \l_tmpf_tl { \int_eval:n { \day } }
 }

\ExplSyntaxOff

% ------------------------------------------------------------------------------------------------ %

\newcommand{\funcExperienceTechnology}[1]
{
    {\textbf{Стек: }
    \itshape
    #1}
}

% Описание рода деятельности компании, к которому я причастен
\newcommand{\funcExperienceAboutCompany}[1]
{
    {\smallskip
    \itshape
    #1
    \smallskip
    \par}
}

\newcommand{\funcExperiencePeriod}[2]
{
    {\monthyearfromdate{#1} -- \monthyearfromdate{#2}} {(\datediffstr{#1}{#2})}
}

\newcommand{\funcExperiencePS}[1]
{
    #1
    \smallskip
}


% ------------------------------------------------------------------------------------------------ %

% СЦ ТТМ
\newcommand{\funcExperienceTtm}
{
    % -------------------------------------------------------------------------------------------- %
    %   Тут идет разбиение компании на 2 должности отдельными функциями, входящими внутрь функции
    %   \cvexperience.
    %
    %   Сделано это для того, чтобы показать изменение должности, но не дублировать компанию и
    %   описание ее деятельности.
    % -------------------------------------------------------------------------------------------- %

    \funcExperiencePeriod{2021-07-10}{сейчас}





    \cvexperienceCompanyLocation
        {
            \href
                {\varExperienceJobUrlTtm}
                {Сервисный центр Транстелематика}
        }
        {г. Саратов}

    % Описание рода деятельности компании, к которому я причастен
    \funcExperienceAboutCompany
    {
        В рамках проекта \href{\varExperienceJobUrlTtmMnt}{\underline{навигационного терминала}}
        для общественного транспорта командная разработка и поддержка Linux сервисов,
        взаимодействующих по шине ZeroMQ посредством Protobuf.
    }

    % -------------------------------------------------------------------------------------------- %
    % Senior C++ developer

    \cvexperiencePosition
        {Senior C++ developer}

    \cvexperiencePeriod
        {Сентябрь 2024 -- \textbf{сейчас} (9 месяцев)}       % Сейчас = июнь 2025

    \smallskip
    \begin{itemize}

        \item
            Разработал для сервиса обновления контента ТС режим master-slave-slave для 3-х
            вагонного трамвая с использованием TFTP и гарантированной синхронизацией контента
            между вагонами.

        \item
            Разработал для сервиса мониторинга оплаты проезда взаимодействие c валидаторами ТС по
            протоколу JsonRPC с использованием sqlite БД.

    \end{itemize}

    % -------------------------------------------------------------------------------------------- %
    % Middle C++ developer

    \cvexperiencePosition
        {Middle C++ developer}

    \cvexperiencePeriod
        {Апрель 2023 -- Сентябрь 2024 (1 год 5 месяцев)}

    \smallskip
    \begin{itemize}

        \item
            Разработал для сервиса информирования пассажиров с ограниченными возможностями
            взаимодействие с новыми устройствами радио информаторов по Ethernet с использованием
            Protobuf.

        \item
            Архитектурно разделил сервис обновления контента на 2 версии работы с внешним API.
            Расширил спектр скачиваемого контента, увеличив количество потоков и HTTP запросов.

        \item
            Разработал для модуля маршрутов алгоритм прогноза прибытия ТС к следующей остановке.

        \item
            Покрыл Unit-тестами \textit{(GTest/GMock)} и документировал \textit{(Doxygen/UML)}
            2 сервиса.

    \end{itemize}

    % -------------------------------------------------------------------------------------------- %
    % Junior C++ developer

    \cvexperiencePosition
        {Junior C++ developer}

    \cvexperiencePeriod
        {Октябрь 2021 -- Апрель 2023 (1 год 6 месяцев)}

    \smallskip
    \begin{itemize}

        \item
            Провел рефакторинг сервиса информирования пассажиров с ограниченными возможностями
            с QT на C++14 с использованием modbus (RS-485).

        \item
            Провел рефакторинг через TDD многопоточного сервиса обновления контента с QT на C++14.
            Использовал libcurl \textit{(HTTP)}, jsoncpp, std::experimental::filesystem, openssl,
            std::thread.

        \item
            Реализовал алгоритм HTTP скачивания контента по чанкам для условий плохого GSM
            соединения и алгоритм возобновления скачивания после полного обрыва связи.

    \end{itemize}

    % -------------------------------------------------------------------------------------------- %

    \funcExperienceTechnology
    {
        C++14, C++17, Python,                                           % Языки
        STL, boost, GTest, QT,                                          % Фреймворки
        CMake,                                                          % Сборка
        Protobuf, ZeroMQ, HTTP, RS485, Ethernet, TFTP, JsonRPC, modbus, % Протоколы
        sqlite,
        Git,                                                            % СКВ
        Bash, Linux.                                                    % Окружение
    }
}


% ------------------------------------------------------------------------------------------------ %

% СГУ - Грант
\newcommand{\funcExperienceSsu}
{
    \cvexperience
        {C++ developer}
        {
            \href
                {\varExperienceJobUrlSsu}
                {СГУ им. Н.Г. Чернышевского}
        }
        {Январь 2023 -- Январь 2025 (2 года)}
        {г. Саратов}

    % Описание рода деятельности компании, к которому я причастен
    \funcExperienceAboutCompany
    {
        Индивидуальная разработка в рамках научного гранта приложения по решению системы
        дифференциальных уравнений, описывающей поведение графена в электрическом поле.
    }

    \begin{itemize}

        \item
            Разработал на С++14 консольное OpenMPI приложение под Linux для кластерных вычислений
            решения системы диф. ур., состоящее из модуля логирования, конфигурации, выбора счетных
            алгоритмов и мониторинга производительности.

        \item
            Увеличил производительность приложения на 40\% по сравнению с версией на языке C,
            написанной до меня, за счет архитектурной оптимизации параллельной части приложения.

    \end{itemize}

    \funcExperiencePS
    {
        Приложение запускалось на 256 ядрах кластера и обсчитывало сетки из миллионов точек.
    }

    \funcExperienceTechnology
    {
        C++14,
        CMake,
        boost::odeint, mpich, jsoncpp,
        SLURM,
        Bash, Git, Linux.
    }
}

% ------------------------------------------------------------------------------------------------ %

% СЭЗ Орджоникидзе
\newcommand{\funcExperienceSez}
{
    \cvexperience
        {Junior C/C++ developer}
        {
            \href
                {\varExperienceJobUrlSez}
                {ОАО СЭЗ им. Серго Орджоникидзе}
        }
        {Октябрь 2019 -- Октябрь 2021 (2 года)}
        {г. Саратов}

    % Описание рода деятельности компании, к которому я причастен
    \funcExperienceAboutCompany
    {
        Разработка авиационной электроники. Embedded разработка под ARM.
        Разработка консольных приложений под Linux.
    }

    \begin{itemize}

        \item
            Разработал на языке C под архитектуру STM32 прототип системы дистанционного управления
            кран-балкой самолета АН-124 по протоколу ZigBee: 2 приемника (на кранах) и 2 передатчика
            (у операторов).

        \item
            Разработал на языке С++ для Ubuntu сниффер для отладки протокола ZigBee через
            UART.

    \end{itemize}

    \funcExperienceTechnology
    {
        C, C++,
        make,
        STM32, UART, ZigBee, JTAG,
        Linux, Git.
    }
}


% ------------------------------------------------------------------------------------------------ %

%
\newcommand{\funcExperiencePrint}
{
    % Заголовок секции
    \cvsection{Опыт работы}

    % Решение проблемы с плотностью слов в строках
    \sloppy

    % Дальше идут повторяющиеся блоки \cvevent + itemize
    %
    % Формат \cvevent:
    %   \cvevent{Должность}{Компания}{Дата начала -- дата окончания}{Местоположение}
    %
    % В itemize перечисляются задачи/успехи


    \funcExperienceTtm          % Middle C/C++ developer в СЦ ТТМ



    \newpage

    \funcExperienceSsu          % C++ developer в СГУ

    \divider

    \funcExperienceSez          % Junior C/C++ developer на СЭЗ
}
