% ------------------------------------------------------------------------------------------------ %
%
% Файл с описанием опыта
%
% ------------------------------------------------------------------------------------------------ %

\newcommand{\varExperienceJobUrlSez}        {https://zaprib.ru/}
\newcommand{\varExperienceJobUrlSsu}        {https://www.sgu.ru/}
\newcommand{\varExperienceJobUrlTtm}        {https://transtelematica.ru/}
\newcommand{\varExperienceJobUrlTtmMnt}     {https://transtelematica.ru/projects/programmnoe-obespechenie/programmnoe-obespechenie-MNT/}



\newcommand{\funcExperienceTechnology}[1]
{
    \textbf{Стек:} #1
}

% Описание рода деятельности компании, к которому я причастен
\newcommand{\funcExperienceAboutCompany}[1]
{
    \textit{#1}
}

% ------------------------------------------------------------------------------------------------ %

%
\newcommand{\funcExperienceTtmDefinition}
{
    В рамках проекта \href{\varExperienceJobUrlTtmMnt}{\underline{навигационного терминала}} для
    общественного транспорта разработка Linux сервисов, взаимодействующих по шине ZeroMQ
    посредством Protobuf.
}


% ------------------------------------------------------------------------------------------------ %

%
\newcommand{\funcExperienceTtmMiddle}
{
    \cvevent
        {Middle C/C++ developer}
        {
            \href
                {\varExperienceJobUrlTtm}
                {Сервисный центр Транстелематика}
        }
        {Апрель 2023 -- \textbf{сейчас} (2 года 2 месяца)}       % Сейчас = июнь 2025
        {г. Саратов}

    % Описание рода деятельности компании, к которому я причастен
    \funcExperienceAboutCompany
    {
        \funcExperienceTtmDefinition
    }

    \begin{itemize}

        \item
            Разработал для сервиса информирования пассажиров с ограниченными возможностями
            взаимодействие с радио информаторами по Ethernet с использованием Protobuf.

        \item
            Разработал для сервиса обновления контента ТС режим master-slave-slave для 3-х
            вагонного трамвая с использованием FTP и гарантированной синхронизацией контента
            между вагонами.

        \item
            Разработал для сервиса мониторинга оплаты проезда взаимодействие c валидаторами ТС по
            протоколу JsonRPC с использованием sqlite БД.

    \end{itemize}

    \funcExperienceTechnology
    {
        C++14, C++17, Python,                                       % Языки
        STL, boost,                                                 % Фреймворки
        CMake,                                                      % Сборка
        Protobuf, ZeroMQ, HTTP, RS485, Ethernet, FTP, JsonRPC       % Протоколы
        sqlite,
        Git,
        Bash, Linux.                                                % Окружение
    }
}


% ------------------------------------------------------------------------------------------------ %

% СГУ - Грант
\newcommand{\funcExperienceSsu}
{
    \cvevent
        {C++ developer}
        {
            \href
                {\varExperienceJobUrlSsu}
                {Саратовский государственный университет им. Н.Г. Чернышевского}
        }
        {Январь 2023 -- Январь 2025 (2 года)}
        {г. Саратов}

    % Описание рода деятельности компании, к которому я причастен
    \funcExperienceAboutCompany
    {
        Разработка в рамках научного гранта приложения по решению диф. ур., описывающей поведение
        графена в электрическом поле.
    }

    \begin{itemize}

        \item
            Разработал консольное OpenMPI приложение под Linux для кластерных вычислений решения
            системы диф. ур.

        \item
            Увеличил производительность приложения на 40\% по сравнению с версией на языке C,
            написанной до меня, за счет архитектурной оптимизации параллельной части приложения.

    \end{itemize}

    Приложение запускалось на 256 ядрах кластера и обладало системой мониторинга производительности.

    \funcExperienceTechnology
    {
        C++14,
        CMake,
        boost, mpich, jsoncpp,
        SLURM,
        Bash, Git, Linux.
    }
}


% ------------------------------------------------------------------------------------------------ %

%
\newcommand{\funcExperienceTtmJunior}
{
    \cvevent
        {Junior C/C++ developer}
        {
            \href
                {\varExperienceJobUrlTtm}
                {Сервисный центр Транстелематика}
        }
        {Октябрь 2021 -- Апрель 2023 (1 год 6 месяцев)}
        {г. Саратов}

    % Описание рода деятельности компании, к которому я причастен
    \funcExperienceAboutCompany
    {
        \funcExperienceTtmDefinition
    }

    \begin{itemize}

        \item
            Переписал сервис информирования пассажиров с ограниченными возможностями
            с QT на C++14 с использованием modbus (RS-485). Покрыл Unit-тестами.

        \item
            Переписал с использованием TDD многопоточный сервис обновления контента ТС с QT на
            C++14. Реализовал алгоритм HTTP скачивания контента по чанкам для условий плохого GSM
            соединения.

    \end{itemize}

    \funcExperienceTechnology
    {
        C++14,                              % Языки
        GTest, QT                           % Фреймворки
        CMake,                              % Сборка
        Protobuf, ZeroMQ, HTTP, modbus,     % Протоколы
        Git,
        Bash, Linux.                        % Окружение
    }
}


% ------------------------------------------------------------------------------------------------ %

% СЭЗ Орджоникидзе
\newcommand{\funcExperienceSez}
{
    \cvevent
        {Junior C/C++ developer}
        {
            \href
                {\varExperienceJobUrlSez}
                {ОАО СЭЗ им. Серго Орджоникидзе}
        }
        {Октябрь 2019 -- Октябрь 2021 (2 года)}
        {г. Саратов}

    % Описание рода деятельности компании, к которому я причастен
    \funcExperienceAboutCompany
    {
        Разработка авиационной электроники. Embedded разработка под ARM.
        Разработка консольных приложений под Linux.
    }

    \begin{itemize}

        \item
            Разработал на языке C под STM32 прототип системы дистанционного управления из 4-х
            ZigBee устройств для кран-балки самолета АН-124.

        \item
            Разработал на языке С++ для Ubuntu сниффер для отладки протокола ZigBee через
            UART.

    \end{itemize}

    \funcExperienceTechnology
    {
        C, C++,
        make,
        STM32, UART, ZigBee,
        Linux, Git.
    }
}


% ------------------------------------------------------------------------------------------------ %

%
\newcommand{\funcExperiencePrint}
{
    % Заголовок секции
    \cvsection{Опыт работы}

    % Решение проблемы с плотностью слов в строках
    \sloppy

    % Дальше идут повторяющиеся блоки \cvevent + itemize
    %
    % Формат \cvevent:
    %   \cvevent{Должность}{Компания}{Дата начала -- дата окончания}{Местоположение}
    %
    % В itemize перечисляются задачи/успехи


    \funcExperienceTtmMiddle    % Middle C/C++ developer в СЦ ТТМ

    \divider

    \funcExperienceSsu          % C++ developer в СГУ

    \divider

    \funcExperienceTtmJunior    % Junior C/C++ developer в СЦ ТТМ

    \divider

    \funcExperienceSez          % Junior C/C++ developer на СЭЗ
}