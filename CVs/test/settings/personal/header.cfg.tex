% ------------------------------------------------------------------------------------------------ %
%
% Файл с настройкой заголовка - личной информации для шапки резюме
%
% ------------------------------------------------------------------------------------------------ %


% Описание информации о кандидате: эти команды-переменные используются только внутри этого файла
% Снаружи используются только команды-функции

\newcommand{\varInfoName}           {Новиков Николай}
\newcommand{\varInfoPosition}       {Middle C++ developer}
\newcommand{\varInfoExperience}     {6 лет опыта}

\newcommand{\varInfoEmail}          {novikov.na.work@yandex.ru}
\newcommand{\varInfoPhone}          {+7(904)243-56-47}

\newcommand{\varInfoTelegramURL}    {https://t.me/}
\newcommand{\varInfoTelegramName}   {StrapiDuStra}

\newcommand{\varInfoGithub}         {strapidustra125}
\newcommand{\varInfoLinkedin}       {novikov-na}
\newcommand{\varInfoOrcid}          {0000-0000-0000-0000}

\newcommand{\varInfoLocation}       {г. Саратов}

\newcommand{\varInfoPhoto}          {photo/9.jpg}

% ------------------------------------------------------------------------------------------------ %

% Команда-функция вывода имени кандидата (первая строка)
\newcommand{\funcInfoSetName}
{
    \name{\varInfoName}
}

% Команда-функция вывода желаемой должности(ей), оно же - Tagline (вторая строка)
\newcommand{\funcInfoSetPosition}
{
    \tagline{\varInfoPosition}
}

% Команда-функция вывода опыта
\newcommand{\funcInfoSetExperience}
{
    \experience{\varInfoExperience}
}

% Команда-функция вывода фотографий по бокам от блока информации со ссылками
% Могут быть две фотографии - слева и/или справа от блока
\newcommand{\funcInfoSetPhotos}
{
    \photoR{3.2cm}{\varInfoPhoto}  % Правое фото
    % \photoL{3.0cm}{\varInfoPhoto}  % Левое фото
}

% Команда-функция генерации блока информации под должностью и именем
% Здесь выбираются нужные поля,а ненужные комментируются
\newcommand{\funcInfoSetTopBlock}
{
    % Объявление пользовательских полей информации
    % Например: \NewInfoField{telegram}{\faIcon{telegram}}[\varInfoTelegramURL]
    % Где,
    %   - первый параметр   - telegram              - имя, по которому функция будет вызываться
    %   - второй параметр   - \faIcon{telegram}     - иконка из пакета fontawesome5
    %   - блок []           - \varInfoTelegramURL   - Начало ссылки, к которой потом допишется PARAM
    %
    % Потом вызывается \telegram{PARAM}
    % Появится иконка с именем, а при наведении будет выводиться конкатенация ссылки из [] и PARAM
    %
    %
    % Или можно вызвать пользовательское поле одной строчкой:
    % \printinfo{символ}{надпись}[префикс ссылки]

    \NewInfoField{telegram}{\faIcon{telegram}}[\varInfoTelegramURL]

    \personalinfo
    {
        % ---------------------------------------------------------------------------------------- %
        %
        %                                       ОЧЕНЬ ВАЖНО!!!
        %
        % Важно позиционирование функций в коде:
        %   1. Если функции слеплены без разделительных строк между ними:
        %       ----------------------------------------
        %       \email      {\varInfoEmail}
        %       \telegram   {\varInfoTelegramName}
        %       \location   {\varInfoLocation}
        %       ----------------------------------------
        %
        %       То они выведутся В СТРОКУ
        %
        %   2. Если добавляются разделительные строки:
        %       ----------------------------------------
        %       \email      {\varInfoEmail}
        %
        %       \telegram   {\varInfoTelegramName}
        %
        %       \location   {\varInfoLocation}
        %       ----------------------------------------
        %
        %       То параметры выведутся в столбик
        %
        %   Можно делить на блоки столбик/строчка, играясь с пустыми строками между функциями!
        %
        % ---------------------------------------------------------------------------------------- %

        % Необходимая информация

        % Жирным - для предпочтительного способа связи
        \textbf
        {
        \email      {\varInfoEmail}
        }
        \telegram   {\varInfoTelegramName}
        \github     {\varInfoGithub}

        % Ненужная информация

        % \location   {\varInfoLocation}
        % \linkedin   {\varInfoLinkedin}
        % \phone      {\varInfoPhone}
        % \orcid      {\varInfoOrcid}
        % \mailaddress{Åddrésş, Street, 00000 Cóuntry}
        % \homepage   {www.homepage.com}

        % \twitter    {@twitterhandle}
        % \xtwitter   {@x-handle}
  }
}

% Команда-функция генерации заголовка с информацией
\newcommand{\funcInfoHeaderPrint}
{
    \funcInfoSetName        % Первая строка - имя кандидата
    \funcInfoSetPosition    % Вторая строка - должность(и)
    \funcInfoSetPhotos      % Фотографии - могут быть две: слева и/или справа от блока информации
    \funcInfoSetExperience  % Опыт работы
    \funcInfoSetTopBlock    % Блок информации под должностью и именем

    \makecvheader           % Функция генерации заголовка: без нее заголовка не будет вообще
}