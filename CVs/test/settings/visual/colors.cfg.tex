% ------------------------------------------------------------------------------------------------ %
% Файл с настройкой цветовой гаммы для документа
%
% ------------------------------------------------------------------------------------------------ %


% Настройка псевдонимов для нестандартных цветов

\definecolor{SlateGrey}         {HTML}  {2E2E2E}
\definecolor{LightGrey}         {HTML}  {666666}
\definecolor{DarkPastelRed}     {HTML}  {450808}
\definecolor{PastelRed}         {HTML}  {8F0D0D}
\definecolor{GoldenEarth}       {HTML}  {E7D192}
\definecolor{RoyalBlue}         {HTML}  {4169E1}
\definecolor{IndianRed}         {HTML}  {CD5C5C}


% Настройка цветов для конкретных секций
\newcommand{\funcColorsConfig}
{
    \colorlet{name}             {black}             % Имя кандидата
    \colorlet{tagline}          {PastelRed}         % Должность или краткое описание
    \colorlet{heading}          {DarkPastelRed}     % Заголовки
    \colorlet{headingrule}      {GoldenEarth}       % Черта под заголовком
    \colorlet{subheading}       {DarkPastelRed}     % Подзаголовки
    \colorlet{accent}           {PastelRed}         % Акценты - названия компаний в \cvevent и \cvexperience
    \colorlet{emphasis}         {SlateGrey}         % Выделение - должности в \cvevent и \cvexperience
    \colorlet{body}             {LightGrey}         % Основной текст
    \colorlet{periods}          {LightGrey}         % Строки с периодом чего-либо (с календариком)
    \colorlet{position}         {black}             % Должность в местах работы

    \colorlet{calendarMarker}   {RoyalBlue}         % Маркер-значок календаря у периодов работы
    \colorlet{locationMarker}   {IndianRed}         % Маркер-значок локации у мест работы
}
