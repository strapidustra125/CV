% ------------------------------------------------------------------------------------------------ %
%
% Файл с настройкой перечислений и маркеров
%
% ------------------------------------------------------------------------------------------------ %



% Команда-метод настройки перечислений и маркеров
\newcommand{\funcEnumsConfig}
{

    % Тут можно выбрать любой размер шрифта всех перечислений в документе: от \tiny до \Huge
    % По умолчанию: \normalsize
    \AtBeginEnvironment{itemize}{\normalsize}


    % Настройка маркеров

    % Размер и стиль маркеров для всех перечислений в документе
    %
    % Размер:
    %   от \tiny до \Huge
    % Стиль:
    %   \textbullet             - Классическая пуля
    %   \textendash	            - Короткое тире
    %   \textasteriskcentered   - Центрированная звёздочка
    %   \textperiodcentered     - Маленькая точка
    %   \textbf{--}             - Жирное тире (ручной вариант)
    \renewcommand{\cvItemMarker}{{\tiny\faBug}}

    % Маркеры для \cvskill (Секция "Навыки или ""Языки")
    \renewcommand{\cvRatingMarker}{\faCircle}

    % Маркер для поля "дата" в \cvevent
    \renewcommand{\cvDateMarker}{\faCalendar*[regular]}

    % Маркер для поля "локация" в \cvevent
    \renewcommand{\cvLocationMarker}{\faMapMarker*}
}