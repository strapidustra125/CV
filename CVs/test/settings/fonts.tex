% ------------------------------------------------------------------------------------------------ %
% Файл с настройкой шрифтов для документа
% Шрифты берутся из отдельной директории
% ------------------------------------------------------------------------------------------------ %

% Пакет для работы со шрифтами
\usepackage{fontspec}

% Команды-переменные для хранения названия единственного шрифта для всего документа

\newcommand{\varFontsGlobal}{PT Sans}
\newcommand{\varFontsGlobalSystemName}{PT_Sans-Web}

% Команды-переменные для основного шрифта для всего документа: латиница, цифры, символы

\newcommand{\varFontsMain}{\varFontsGlobal}
\newcommand{\varFontsMainSystemName}{\varFontsGlobalSystemName}

% Команды-переменные для шрифта без засечек

\newcommand{\varFontsSans}{\varFontsGlobal}
\newcommand{\varFontsSansSystemName}{\varFontsGlobalSystemName}

% Команды-переменные для символов кириллицы

\newcommand{\varFontsCyrillic}{\varFontsGlobal}
\newcommand{\varFontsCyrillicSystemName}{\varFontsGlobalSystemName}

% Команды-переменные для файлов шрифтов

\newcommand{\varFontsDirectory}{../../fonts/}
\newcommand{\varFontsExtension}{.ttf}


% Команда-функция настройки основного шрифта
\newcommand{\funcFontsSetMain}[4]
{
  % Установка основного шрифта
  \setmainfont{#1}                % Название основного шрифта (может не совпадать с именем в файловой системе) (параметр 1)
  [
    Path=#2,                      % Путь до директории файла шрифта (параметр 2)
    Extension=#3,                 % Расширение файла шрифта с точкой (параметр 3)
    UprightFont=#4,               % Имя шрифта в файловой системе без расширения (параметр 4)
    BoldFont=#4-Bold,             % Имя жирного шрифта
    ItalicFont=#4-Italic,         % Имя наклонного шрифта
    BoldItalicFont=#4-BoldItalic  % Имя жирно-наклонного шрифта
  ]
}

% Команда-функция настройки шрифта без засечек
\newcommand{\funcFontsSetSans}[4]
{
  % Установка основного шрифта без засечек
  \setsansfont{#1}                % Название основного шрифта (может не совпадать с именем в файловой системе) (параметр 1)
  [
    Path=#2,                      % Путь до директории файла шрифта (параметр 2)
    Extension=#3,                 % Расширение файла шрифта с точкой (параметр 3)
    UprightFont=#4,               % Имя шрифта в файловой системе без расширения (параметр 4)
    BoldFont=#4-Bold,             % Имя жирного шрифта
    ItalicFont=#4-Italic,         % Имя наклонного шрифта
    BoldItalicFont=#4-BoldItalic  % Имя жирно-наклонного шрифта
  ]
}

% Команда-функция настройки шрифта кириллицы
\newcommand{\funcFontsSetCyrillic}[4]
{
  % Установка основного шрифта для кириллицы
  \newfontfamily\cyrillicfont{#1} % Название основного шрифта (может не совпадать с именем в файловой системе) (параметр 1)
  [
    Path=#2,                      % Путь до директории файла шрифта (параметр 2)
    Extension=#3,                 % Расширение файла шрифта с точкой (параметр 3)
    UprightFont=#4,               % Имя шрифта в файловой системе без расширения (параметр 4)
    BoldFont=#4-Bold,             % Имя жирного шрифта
    ItalicFont=#4-Italic,         % Имя наклонного шрифта
    BoldItalicFont=#4-BoldItalic  % Имя жирно-наклонного шрифта
  ]
}

% Основная команда-функция для настройки всех шрифтов документа
\newcommand{\funcFontsSet}
{
  \usepackage{polyglossia}

  \setmainlanguage{russian}
  \setotherlanguage{english}

  % Настройка шрифтов: основной, без засечек и кириллицу

  \funcFontsSetMain{\varFontsMain}{\varFontsDirectory}{\varFontsExtension}{\varFontsMainSystemName}
  \funcFontsSetSans{\varFontsSans}{\varFontsDirectory}{\varFontsExtension}{\varFontsSansSystemName}
  \funcFontsSetCyrillic{\varFontsCyrillic}{\varFontsDirectory}{\varFontsExtension}{\varFontsCyrillicSystemName}

  % Выбор шрифта по умолчанию:
  %  - \rmdefault - с засечками (serif)         - Roman
  %  - \sfdefault - без засечек (sans-serif)    - Sans-serif
  %  - \ttdefault - моноширинный (typewriter)   - Teletype

  \renewcommand{\familydefault}{\rmdefault}

  % Настройка размера шрифтов для конкретных полей
  %
  % | Команда         | Относительный размер | Описание                                |
  % | --------------- | -------------------- | --------------------------------------- |
  % | \tiny           | Очень маленький      | Самый мелкий размер                     |
  % | \scriptsize     | Очень маленький      | Чаще всего для подписей под формулами   |
  % | \footnotesize   | Маленький            | Для сносок, подвалов, мелких деталей    |
  % | \small          | Чуть меньше обычного | Используется для второстепенного текста |
  % | \normalsize     | Обычный размер       | Размер по умолчанию                     |
  % | \large          | Чуть больше обычного | Для выделения, но не заголовков         |
  % | \Large          | Ещё больше           | Часто используется в заголовках         |
  % | \LARGE          | Большой              | Подзаголовки или крупный текст          |
  % | \huge           | Очень большой        | Реже используется, например в титуле    |
  % | \Huge           | Самый большой        | Максимальный стандартный размер         |

  % Настройка стиля шрифтов
  %
  % | Команда     | Что делает (весь текст после неё) |
  % | ----------- | --------------------------------- |
  % | \rmfamily   | Шрифт с засечками                 |
  % | \sffamily   | Без засечек                       |
  % | \ttfamily   | Моноширинный                      |
  % | \bfseries   | Жирный                            |
  % | \mdseries   | Обычный вес (medium)              |
  % | \itshape    | Курсив                            |
  % | \slshape    | Наклонный (slanted)               |
  % | \scshape    | Малые заглавные                   |
  % | \upshape    | Вертикальный, не наклонный текст  |


  \renewcommand{\namefont}        {\huge\rmfamily\bfseries}   % Имя кандидата
  \renewcommand{\personalinfofont}{\footnotesize}             % Инфо кандидата: телефон, почта и т.д.
  \renewcommand{\cvsectionfont}   {\LARGE\rmfamily\bfseries}  % Заголовки секций резюме
  \renewcommand{\cvsubsectionfont}{\large\bfseries}           %
}
