% ------------------------------------------------------------------------------------------------ %
%
% Файл с настройкой заголовка - личной информации для шапки резюме
%
% ------------------------------------------------------------------------------------------------ %


% Описание информации о кандидате: эти команды-переменные используются только внутри этого файла
% Снаружи используются только команды-функции

\newcommand{\varInfoName}           {Новиков Николай}
\newcommand{\varInfoPosition}       {C++ developer}
\newcommand{\varInfoEmail}          {}
\newcommand{\varInfoTelegramURL}    {https://t.me/}
\newcommand{\varInfoTelegramName}   {StrapiDuStra}
\newcommand{\varInfoPhone}          {+7(904)243-56-47}
\newcommand{\varInfoGithub}         {strapidustra125}
\newcommand{\varInfoLocation}       {Саратов}

\newcommand{\varInfoPhoto}          {photo/photo.png}

% Команда-функция вывода имени кандидата (первая строка)
\newcommand{\funcInfoSetName}
{
    \name{\varInfoName}
}

% Команда-функция вывода желаемой должности(ей), оно же - Tagline (вторая строка)
\newcommand{\funcInfoSetPosition}
{
    \tagline{\varInfoPosition}
}

% Команда-функция вывода фотографий по бокам от блока информации со ссылками
% Могут быть две фотографии - слева и/или справа от блока
\newcommand{\funcInfoSetPhotos}
{
    % \photoL{2.8cm}{\varInfoPhoto}  % Левое фото
    % \photoR{2.8cm}{\varInfoPhoto}  % Правое фото
}

% Команда-функция генерации блока информации под должностью и именем
% Здесь выбираются нужные поля,а ненужные комментируются
\newcommand{\funcInfoSetTopBlock}
{
    % Объявление пользовательских полей информации
    % Например: \NewInfoField{telegram}{\faIcon{telegram}}[\varInfoTelegramURL]
    % Где,
    %   - первый параметр   - telegram              - имя, по которому функция будет вызываться
    %   - второй параметр   - \faIcon{telegram}     - иконка из пакета fontawesome5
    %   - блок []           - \varInfoTelegramURL   - Начало ссылки, к которой потом допишется PARAM
    %
    % Потом вызывается \telegram{PARAM}
    % Появится иконка с именем, а при наведении будет выводиться конкатенация ссылки из [] и PARAM
    %
    %
    % Или можно вызвать пользовательское поле одной строчкой:
    % \printinfo{символ}{надпись}[префикс ссылки]

    \NewInfoField{telegram}{\faIcon{telegram}}[\varInfoTelegramURL]

    \personalinfo
    {
        % Необходимая информация

        \email      {\varInfoEmail}
        \telegram   {\varInfoTelegramName}
        \location   {\varInfoLocation}

        % Опциональная информация

        \phone      {\varInfoPhone}
        \github     {\varInfoGithub}

        % Ненужная информация

        % \mailaddress{Åddrésş, Street, 00000 Cóuntry}
        % \orcid{0000-0000-0000-0000}
        % \linkedin{your_id}
        % \homepage{www.homepage.com}
        % \twitter{@twitterhandle}
        % \xtwitter{@x-handle}
  }
}