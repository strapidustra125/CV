\documentclass[a4paper,12pt]{article}

\usepackage[T2A]{fontenc}
\usepackage{epigraph}
\usepackage[english]{babel} % языковой пакет
\usepackage{graphicx} % для картинок
\usepackage{amsmath,amsfonts,amssymb} %математика
\usepackage{mathtools}


\begin{document}
\section{Первая часть}\label{sec:s1}
Однажды Эрнест Хемингуэй поспорил, что сможет написать самый короткий рассказ, способный растрогать любого. Он выиграл спор:\\

\epigraph{Как там с деньгами обстоит вопрос?}{Терентьев Михал Палыч}

Параграф для \textbf{жирного},
\textit{курсивного}, \underline{подчёркнутого}
текста. Хотелось бы отметить, что теперь уже будет отступ, в отличие от первой строки первой части.

\[
\sum{\frac{(\mu - \bar{x})^2}{n-1}}
\]

\section{Вторая часть}
Ссылка на \textit{\ref{sec:s1}} часть
\end{document}